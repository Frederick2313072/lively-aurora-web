% !TEX TS-program = xelatex
% !TEX encoding = UTF-8 Unicode
% !Mode:: "TeX:UTF-8"

\documentclass{resume}
\usepackage{zh_CN-Adobefonts_external} % Simplified Chinese Support using external fonts (./fonts/zh_CN-Adobe/)
%\usepackage{zh_CN-Adobefonts_internal} % Simplified Chinese Support using system fonts
\usepackage{linespacing_fix} % disable extra space before next section
\usepackage{cite}
% 删除geometry包重复加载,因为resume.cls已经加载了
% \usepackage{enumitem} % 用于调整列表间距 - 暂时禁用以避免冲突
% % 设置列表格式
% \setlist[itemize]{leftmargin=*,topsep=2pt,itemsep=1pt,parsep=0pt}

\hypersetup{colorlinks=true, urlcolor=blue, linkcolor=blue}

\begin{document}
\pagenumbering{gobble} % suppress displaying page number

\name{徐媛}

\basicInfo{
  \email{2313072@mail.nankai.edu.cn} 
  \phone{(+86) 18918007833} 
  \linkedin[Github]{https://github.com/Frederick2313072}
  \location{天津,中国}
}

\vspace{-2ex}
\section{教育背景}
\datedsubsection{\textbf{南开大学}, 天津}{2022 -- 2026}
\textit{本科}\ 统计与数据科学

\vspace{-1.8ex}
\section{实习经历}
\datedsubsection{\textbf{极欧云 | AlAgent 全栈开发实习生}} {2025.04 -- 2025.07}

\textbf{项目描述:}

参与公司核心产品 SalesAgent 与 RAG 智能问答系统的功能开发与性能优化,项目旨在构建具备上下文记忆、多轮对话能力的企业级智能助手,服务于销售自动化、客户支持场景。

\textbf{个人工作:}
\begin{itemize}
    \item 建立了SalesAgent的综合回归测试框架,开发自动化脚本覆盖核心工作流程和边缘情况,成功识别并解决了8个关键缺陷。
    \item 通过实验三种检索机制优化了Memory模块——带时间权重的嵌入、基于查询的聚类和动态摘要,在长对话一致性上提升了18\%。
    \item 通过重新设计文档分块策略(语义+结构视角)并从稀疏检索升级到混合检索(BM25+向量融合)协助增强RAG(检索增强生成)系统,问答准确率提升了15\%。
    \item 解决了多轮对话中的上下文漂移和信息冗余等挑战,平衡准确率、响应速度和内存使用以确定最优检索策略。

    \item 通过构建高质量分块同时保持语义完整性,提升了长文档RAG模块的信息召回率。

\end{itemize}

\datedsubsection{\textbf{Dify开源平台 | LLM应用与RAG生态实习工程师}} {2025.04 -- 2025.07}

\textbf{项目描述:}

参与 Dify 开源平台的插件与数据接入生态建设,围绕 Agent 工具链、RAG 数据管道与图谱检索能力迭代,提升企业级场景下的可用性与扩展性。

\textbf{个人工作:}
\begin{itemize}
    \item 为v2ex、linuxdo和GitHub开发并发布了社区插件,标准化查询参数和认证,支持关键词搜索、主题聚合和速率限制以进行实时Agent检索和摘要。
    \item 设计并实现了Azure Blob和OneDrive数据源集成与增量同步,支持OAuth/密钥双通道认证、目录和权限映射、分块上传和可恢复传输,实现向量化摄取。
    \item 维护和优化了Pinecone向量检索插件,支持批量写入、命名空间、元数据过滤、并发和重试策略,通过增强分块和去重提升召回质量和延迟稳定性。
    \item 使用Neo4j工程化实体关系建模和Cypher查询范式,探索Graph-RAG解决方案以基于关系约束进行证据检索和答案生成。

\end{itemize}

\vspace{-1.8ex}
\section{项目经历}
\datedsubsection{\textbf{PeerPortal - 去中心化留学信息平台}} {2025年7月 -- 至今}
\role{全栈开发工程师}{\href{https://github.com/PeerPortal/web}{Code}}

\textbf{项目描述:}
一个集成了智能对话、实时消息、论坛交流、文件管理和精准匹配的去中心化留学双边信息平台。项目采用现代化全栈架构,为申请者和引路人提供全方位、个性化的留学申请指导服务,打破传统信息壁垒。

\textbf{技术架构:}
\begin{itemize}
    \item \textbf{前端:}基于 Next.js 15 + React 19 + TypeScript 构建现代化SPA,采用 Tailwind CSS + Radix UI 组件库实现响应式设计,使用 Zustand 进行状态管理。
    \item \textbf{后端:}采用 FastAPI + Python 构建高性能异步API服务,集成 Supabase(PostgreSQL) 作为主数据库,支持 WebSocket 实时通信。
    \item \textbf{基础设施:}Docker 容器化部署,支持文件上传存储、用户认证、数据库优化等企业级功能。
\end{itemize}

\textbf{个人工作与职责:}
\begin{itemize}
    \item 独立设计了全栈系统架构,构建了包含9个核心业务模块的综合平台,涵盖用户管理、论坛、实时消息、文件上传和智能匹配。
    \item 使用Next.js 15 App Router开发了现代化SPA前端,实现用户认证、个人资料管理、论坛讨论和实时聊天,实现了60\%的组件复用率。
    \item 使用FastAPI构建了包含50+个RESTful API端点的强大后端,涵盖用户、论坛、消息和文件上传功能,API响应时间保持在100ms以内。
    \item 设计了21表数据库架构,通过索引策略和查询优化将数据库响应时间提升30\%。
    \item 实现了基于WebSocket的实时通信系统,支持一对一导师-学生聊天,消息状态管理和在线状态,延迟在50ms以内。
    \item 设计了三层角色-权限-资源模型,通过JWT无状态认证和中间件实现细粒度API访问控制,确保数据安全和隐私。
    \item 支持1000+并发连接,通过异步编程、连接池管理和消息队列实现99.9\%消息投递成功率。

\end{itemize}

\datedsubsection{\textbf{NKUWiki知识图谱构建(LLM驱动的校园问答)}} {2024年10月 -- 至今}
\role{ETL架构师 \& 安全工程师}{\href{https://github.com/NKU-WIKI/nkuwiki}{Code}}

\textbf{项目描述:}
一个基于RAG检索增强生成的南开大学校园知识共享平台,采用"开源·共治·普惠"理念构建南开知识共同体。项目处理数据量10GB+,覆盖小红书、知乎、校园集市3000+帖子,数据清洗准确率96\%,结构化处理效率提升60\%。基于LangChain优化RAG问答,构建50万+token知识库,HuggingFace嵌入模型检索召回率91\%,问答响应时间优化至1.2s,准确率达到82\%。

\textbf{个人工作与职责:}
\begin{itemize}
    \item 设计并实现了支持异构数据源的分布式爬虫系统,利用Playwright + Selenium进行现代SPA和传统网页抓取,复杂DOM处理性能提升40\%。
    \item 对校园市场进行深度逆向工程分析,破解API签名算法(HMAC-MD5、时间戳验证),逆向JavaScript加密算法实现动态签名生成,数据获取成功率达95\%。。
    \item 实现动态User-Agent轮换、IP代理池管理和请求频率控制,通过浏览器指纹混淆实现99.5\%爬虫正常运行时间和每日10万条记录的数据收集。
    \item 设计异步ETL流水线集成Qdrant向量数据库、Elasticsearch和MySQL,实现增量数据同步,处理延迟在100ms以内。
\end{itemize}

\vspace{-2ex}
\section{专业技能}
\textbf{前端开发}
    \begin{itemize}
        \item 熟悉现代前端技术栈:React 19、Next.js 15、TypeScript 5,具备构建大型SPA应用的能力。
        \item 熟练使用UI框架:Tailwind CSS、Radix UI、MUI,擅长响应式设计和基于组件的开发。
        \item 掌握状态管理:Zustand、Redux;前端工程工具:Webpack、Vite、Turbopack
        \item 熟悉 Docker 容器化技术及相关生态,能够独立完成服务的打包、部署与维护。
    \end{itemize}
    
\textbf{后端开发}
    \begin{itemize}
        \item 掌握 Python,运用 FastAPI 构建高性能、异步化的 RESTful API 及 WebSocket 服务。
        \item 掌握数据库技术:PostgreSQL、MySQL,具备数据库设计、查询优化和索引策略制定能力。
        \item 熟悉云服务平台:Supabase、Vercel,以及 Docker 容器化技术和CI/CD部署流程。
        \item 了解 Django、Spring Boot 等其他后端框架,具备跨技术栈开发能力。
    \end{itemize}



\textbf{AI应用开发}
    \begin{itemize}
        \item 掌握 LangChain 框架,具备独立设计与开发 RAG 检索增强生成及 Agent 应用的能力。
        \item 熟悉 ReAct, Self-ask 等 agent 思维框架,理解 Agent 状态管理、工具调用与多模态交互的核心机制。
        \item 拥有 Prompt Engineering 实践经验,擅长设计、优化与评估提示以激发模型最佳性能。
        \item 了解模型Fine-tuning流程,有使用 ModelScope(SWIFT 框架进行lora 微调) 等平台进行模型训练的经验。
        \item 具备 AI 应用的前后端集成经验,包括流式响应、实时交互等技术实现。
    \end{itemize}



\end{document}